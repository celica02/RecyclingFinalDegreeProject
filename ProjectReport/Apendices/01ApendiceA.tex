%---------------------------------------------------------------------
%
%                          Ap�ndice 1
%
%---------------------------------------------------------------------

\chapter{Materiales}
\label{ap1:Materiales}

%\begin{FraseCelebre}
%\begin{Frase}
%...
%\end{Frase}
%\begin{Fuente}
%...
%\end{Fuente}
%\end{FraseCelebre}

%\begin{resumen}
%...
%\end{resumen}

%-------------------------------------------------------------------
\section{Materiales adjuntos}
%-------------------------------------------------------------------
\label{ap1:intro}

Con la memoria se han entregado de manera adjunta en la carpeta de Google Drive dos archivos con extensi�n \textit{.apk} que corresponden a las dos aplicaciones m�viles desarrolladas en este Trabajo de Fin de Grado. Puede realizarse la instalaci�n de ambas para probar su funcionamiento sobre objetos reales.

El resto de materiales desarrollados pueden encontrarse en el repositorio de GitHub \url{https://github.com/celica02/RecyclingFinalDegreeProject}. Tanto el c�digo de las aplicaciones como los modelos ya entrenados y las im�genes utilizadas para ello se encuentran as� almacenadas en sus correspondientes carpetas. Toda la informaci�n del repositorio puede encontrarse en el archivo \texttt{Readme} de este mismo.


% Variable local para emacs, para  que encuentre el fichero maestro de
% compilaci�n y funcionen mejor algunas teclas r�pidas de AucTeX
%%%
%%% Local Variables:
%%% mode: latex
%%% TeX-master: "../Tesis.tex"
%%% End:
