%---------------------------------------------------------------------
%
%                          Cap�tulo 5
%
%---------------------------------------------------------------------

\chapter{Aplicaci�n de Identificaci�n de Objetos}

\begin{resumen}
En este cap�tulo se habla del desarrollo de la aplicaci�n m�vil. Esta aplicaci�n tiene como objetivo identificar y diferenciar diversos objetos y residuos que una vez identificado se explica al usuario cu�l es la forma adecuada de desecharlos y reciclarlos. La aplicaci�n es para dispositivos con sistema operativo Android y est� llevada a cabo en Android Studio con Java y la librer�a de TensorFlow Lite para introducir la identificaci�n. 

\end{resumen}

%-------------------------------------------------------------------
\section{Tensorflow Lite}
%-------------------------------------------------------------------
\label{cap5:sec:tensorflow-lite}
TensorFlow Lite no s�lo facilita el trabajo de entrenar la red neuronal, sino tambi�n el uso del modelo generado en la aplicaci�n. 

Cuenta con dos librer�as 


 Para este proyecto se decidi� aprovechar el ejemplo que 

%-------------------------------------------------------------------
\section{Identificaci�n de objetos}
%-------------------------------------------------------------------
\label{cap5:sec:identificacion-objetos}
\todo{Qu� hace -> apuntas a un objeto saca con qu� porcentaje considera que es cierto material.}




...

%-------------------------------------------------------------------
%\section*{\NotasBibliograficas}
%-------------------------------------------------------------------
%\TocNotasBibliograficas

%Citamos algo para que aparezca en la bibliograf�a\ldots
%\citep{ldesc2e}

%\medskip

%Y tambi�n ponemos el acr�nimo \ac{CVS} para que no cruja.

%Ten en cuenta que si no quieres acr�nimos (o no quieres que te falle la compilaci�n en ``release'' mientras no tengas ninguno) basta con que no definas la constante \verb+\acronimosEnRelease+ (en \texttt{config.tex}).


%-------------------------------------------------------------------
%\section*{\ProximoCapitulo}
%-------------------------------------------------------------------
%\TocProximoCapitulo

...

% Variable local para emacs, para  que encuentre el fichero maestro de
% compilaci�n y funcionen mejor algunas teclas r�pidas de AucTeX
%%%
%%% Local Variables:
%%% mode: latex
%%% TeX-master: "../Tesis.tex"
%%% End:
