%---------------------------------------------------------------------
%
%                          Cap�tulo 5
%
%---------------------------------------------------------------------

\chapter{Aplicaci�n de Identificaci�n de Objetos}



%-------------------------------------------------------------------
\section{Introducci�n}
%-------------------------------------------------------------------
\label{cap5:sec:introduccion}

Se trata de una aplicaci�n m�vil para Android a trav�s de la cual, apuntando un objeto se identifica el material principal de este y se informa al usuario de c�mo se deber�a reciclar el material identificado.


%-------------------------------------------------------------------
\section{Tensorflow Lite}
%-------------------------------------------------------------------
\label{cap5:sec:tensorflow-lite}
Por qu�, c�mo se usa, qu� cosas proporciona/facilita.

%-------------------------------------------------------------------
\section{Identificaci�n de Objetos}
%-------------------------------------------------------------------
\label{cap5:sec:identificacion-objetos}
Qu� hace -> apuntas a un objeto saca con qu� porcentaje considera que es cierto material.
�Dice d�nde se recicla? No s� si se llegar� a hacer.




...

%-------------------------------------------------------------------
%\section*{\NotasBibliograficas}
%-------------------------------------------------------------------
%\TocNotasBibliograficas

%Citamos algo para que aparezca en la bibliograf�a\ldots
%\citep{ldesc2e}

%\medskip

%Y tambi�n ponemos el acr�nimo \ac{CVS} para que no cruja.

%Ten en cuenta que si no quieres acr�nimos (o no quieres que te falle la compilaci�n en ``release'' mientras no tengas ninguno) basta con que no definas la constante \verb+\acronimosEnRelease+ (en \texttt{config.tex}).


%-------------------------------------------------------------------
%\section*{\ProximoCapitulo}
%-------------------------------------------------------------------
%\TocProximoCapitulo

...

% Variable local para emacs, para  que encuentre el fichero maestro de
% compilaci�n y funcionen mejor algunas teclas r�pidas de AucTeX
%%%
%%% Local Variables:
%%% mode: latex
%%% TeX-master: "../Tesis.tex"
%%% End:
