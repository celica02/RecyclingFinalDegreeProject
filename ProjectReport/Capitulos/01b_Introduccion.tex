%---------------------------------------------------------------------
%
%                          Cap�tulo 1
%
%---------------------------------------------------------------------

\chapter{Introduction}

%\begin{FraseCelebre}
%\begin{Frase}
%\end{Frase}
%\begin{Fuente}

%\end{Fuente}
%\end{FraseCelebre}

%\begin{resumen}

%\end{resumen}


%-------------------------------------------------------------------
%\section{Introducci�n}
%-------------------------------------------------------------------
%\label{cap1:sec:introduccion}


%-------------------------------------------------------------------
\section{Motivation}
%-------------------------------------------------------------------
\label{cap1b:sec:motivation}

Waste generation is linked to the current development model of society and constitutes one of the main environmental problems the world is facing~\citep{sanchez2007gestion}. Waste can be classified into two categories: those produced by industrial activity, called industrial waste, and those generated by human activity itself, called urban waste.

A large part of the pollution and $\mathrm{CO_2}$ emissions come from large companies. For example, in 2018 25\% of emissions in Spain were generated by only ten companies~\citep{Emergencia2019}. However, this coursework will focus on urban waste, which is where population can mainly be aware of and take measures in this regard.

Urban solid waste problem comes from recent years increment use of non-return packaging. These packages can be made of different materials such as cellulose, glass, plastic or mixed (laminated paper, laminated fabrics, etc.) which complicates their treatment, since a previous selection and separation must be done~\citep{fonfria1989ingenieria}.

Poor waste management can cause irreversible environmental impacts. Nowadays, it can already be seen many of the effects that seemed to come in the future~\cite{futureclimatechange}. Among them, it can be highlighted the increasing temperatures around the globe, the disappearance of glaciers (both in mountains and in the polar ice caps~\cite{caballero2007efecto}) or the decrease in 68\% of vertebrates' population~\cite{report2020WWF}.

In Spain, temperatures have increased throughout both national territory and the Mediterranean, among its sea level increment. Also, dilation of summer around 9 days per decade, which means that it is currently 5 weeks longer than at the beginning of the eighties. Some other effects are the disappearance of more than half of the Spanish glaciers and changes in the distribution, behaviour and nutrition of biodiversity, among other factors~\citep{Emergencia2019}.


But proper waste management is something within the reach of any citizen.

Mixing materials is not only considered polluting because it makes their recovery and recycling difficult. Waste separation is a costly and polluting process which not always have satisfactory results. This is due to the fact that, sometimes, the recovered materials are low-quality made and come with tons of non-recyclable particles because they have been combined with other material kinds. Therefore, to facilitate this process we find special containers for each type of waste.

Separating waste properly is a great contribution to take care of the environment. As discussed above, there are a lot of different materials and types of waste, which sometimes can be difficult and confusing knowing how they should be separated correctly. Thus, due to this problem, the motivation for this project comes up. The main objective is to develop an object identification application that, using the camera of a mobile phone, indicates the appropriate way to dispose an identified waste.
With this application, it would be easy for everyone to solve in a comfortable way any doubts about recycling that may appear, promoting this way proper eco-friendly attitudes.



%-------------------------------------------------------------------
\section{Objetives}

%-------------------------------------------------------------------
\label{cap1b:sec:objetives}

The main objective of the project is the development of a mobile application focused on recycling. This app identifies objects and gives information about the material and the appropriate way to recycle them. Through it, users will be able to identify different wastes to solve quickly and easily any doubts about how dispose them correctly.

To achieve its correct behaviour artificial vision techniques are needed, which currently involve the use of trained neural networks. To do so, it is necessary to have a large number of correctly labeled waste images (known as dataset) that are used to ``teach'' (train) the neural network. Obtaining said dataset becomes one of the main sub-objectives of the project. In order to facilitate the process of obtaining the images, it is decided to develop a secondary application. The objective of this one is to avoid the tedious and slow process that obtaining images can be. It works as a computer application that generates synthetic images from three-dimensional models. The application generates as many synthetic images, from the available models, as desired to train the neural network.



%-------------------------------------------------------------------
\section{Tools}
%-------------------------------------------------------------------
\label{cap1b:sec:tools}

Android Studio has been used to develop the identification mobile application. To train and obtain the model necessary for the application performance, a script was coded in Python using Numpy and Tensorflow Lite libraries and PyScripter as the editor.

The image generator application was developed using the video game engine Unity. Every three-dimensional model used in this application have been sourced from CGTrader\footnote{\url{https://www.cgtrader.com/}}, Free3D\footnote{\url{https://free3d.com/es/}}, TurboSquid\footnote{\url{https://www.turbosquid.com/}}, 3DModelHeaven\footnote{\url{https://3dmodelhaven.com/}} and Unity Asset Store\footnote{\url{https://assetstore.unity.com/}}.

Lastly, for the project's version control GitHub\footnote{\url{https://github.com/celica02/RecyclingFinalDegreeProject}} has been used and for the project report development, the \texis template in TexMaker editor was chosen.

The entire process has been developed on a Windows device and the aplication tests were made on an Android mobile device.



%-------------------------------------------------------------------
\section{Work plan}
%-------------------------------------------------------------------
\label{cap1b:sec:work-plan}

The coursework is divided into three different parts: the image generation, training the neural network and develop the mobile object identification application.

The first one is focused on the develop of an application that loads 3D models and makes numerous images for each one, changing their position, rotation and background to obtain diversity in the images. The models are organised by their material so the images generated must be arranged the same way.
The application development is divided into two iterations: the first one is where all the main functionalities are developed and simultaneously being verified. Meanwhile, in the second iteration takes place the dataset generation.

Once the images are generated, the neural network traning takes place, which is the second part of the project. To develop this, it is necessary to investigate the different types of neural networks and choose the most appropriate option to be used in a mobile application. In addition, it will be necessary to obtain some datasets made of real images of the materials selected. Then, make tests and comparisons to check the accuracy in different cases.

Finally, the development of the mobile devices application continues, using the trained model to identify the material of the object that is being focused on with the camera. Besides the material, it is also indicated how it has to be recycled.
At this point in the coursework various tests must be done to verify the application performance. To do so, the different trained models' accuracy will be compared using diverse everyday objects.




%-------------------------------------------------------------------
%\section*{\NotasBibliograficas}
%-------------------------------------------------------------------
%\TocNotasBibliograficas

%Citamos algo para que aparezca en la bibliograf�a\ldots
%\citep{ldesc2e}

%\medskip


%-------------------------------------------------------------------
%\section*{\ProximoCapitulo}
%-------------------------------------------------------------------
%\TocProximoCapitulo