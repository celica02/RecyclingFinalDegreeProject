%---------------------------------------------------------------------
%
%                          Cap�tulo 1
%
%---------------------------------------------------------------------

\chapter{Introduction}

%\begin{FraseCelebre}
%\begin{Frase}
%\end{Frase}
%\begin{Fuente}

%\end{Fuente}
%\end{FraseCelebre}

%\begin{resumen}

%\end{resumen}


%-------------------------------------------------------------------
%\section{Introducci�n}
%-------------------------------------------------------------------
%\label{cap1:sec:introduccion}


%-------------------------------------------------------------------
\section{Motivation}
%-------------------------------------------------------------------
\label{cap1b:sec:motivation}

The waste generation is linked to the current development model of society and constitutes one of the main environmental problems the world is facing~\citep{sanchez2007gestion}. Waste can be classified into two types: those produced by industrial activity, called industrial waste, and those generated by human activity itself, called urban waste.

A large part of the pollution and emissions of $\mathrm{CO_2}$ come from large companies, for example in 2018 25\% of the emissions in Spain were generated by only ten companies~\citep{Emergencia2019}. Even though, this coursework will focus on urban waste, which is where the population can be aware of and take measures in this regard.

The problem of urban solid waste comes from recent years increment use of non-return packaging. These packages can be made of different materials such as cellulose, glass, plastic or mixed (laminated paper, laminated fabrics, etc.) which complicates their treatment, since a previous selection and separation must be done~\citep{fonfria1989ingenieria}.

Poor waste management can cause irreversible environmental impacts. Nowadays it can already be see many of the effects that seemed to come in the future~\cite{futureclimatechange}. Among them it can be highlighted the increasing temperatures around the globe, the disappearance of glaciers (both in the mountains and in the polar ice caps~\cite{caballero2007efecto}) or the decrease in 68\% of the population of vertebrates~\cite{report2020WWF}.

In Spain, the temperatures have increase throughout the national territory and the Mediterranean, among its sea level increment. As well the dilation of summer around 9 days per decade, which means that currently it's 5 weeks longer than at the beginning of the eighties. Other effects are the disappearance of more than half of the spanish glaciers and changes in the distribution, behavior and nutrition of biodiversity, among other factors~\citep{Emergencia2019}.


But proper waste management is something within the reach of any citizen.

Mixing materials is not only polluting because it makes their recovery and recycling difficult. Waste separation is a costly and polluting process which not always have satisfactory results. This is due to the fact that sometimes the recovered materials are low quality and with a high level of non-recyclable particles because they have been combined with others. Therefore, to facilitate this process we find special containers for each type of waste.

Separating waste properly is a great contribution to take care of the environment. As discussed above, there are a lot of different materials and types of waste, which sometimes can be difficult and confusing knowing how they should be separated correctly. Due to this difficulty arises the motivation for this final degree project. The main objective is to develop an object identification application that, using the camera of a mobile phone, indicates the appropriate way to dispose an identified waste.
With this application, it would be easy for citizen to solve in a comfortable way any doubts about recycling that may arise, promoting this way eco-friendly attitudes.



%-------------------------------------------------------------------
\section{Objetives}

%-------------------------------------------------------------------
\label{cap1b:sec:objetives}

The main objective of the project is the development of a mobile application focused on recycling. This app identifies objects and gives information about the material and the appropriate way to recycle them. Through it, users will be able to identify different wastes to solve quickly and easily any doubts about how dispose them correctly.

To achieve its correct behaviour it is necessary to use artificial vision techniques, which currently involve the use of trained neural networks. To do so, it is necessary to have a large number of correctly labeled waste images (known as dataset) that are used to ``teach'' (train) the neural network. Obtaining said dataset becomes one of the main sub-objectives of the project. In order to facilitate the process of obtaining the images, is decided to develop a second application. The objective of this one is to avoid the tedious and slow process that obtaining images can be. This one is a computer application that generates synthetic images from three-dimensional models. This application generates as many synthetic images, from the available models, as desired to train the neural network.



%-------------------------------------------------------------------
\section{Tools}
%-------------------------------------------------------------------
\label{cap1b:sec:tools}

Android Studio it has been used to develop the identification mobile application. To train and obtain the model necessary for the application performance, it was created a script in Python using Numpy and Tensorflow Lite libraries and PyScripter as the editor. This first part has been based on the examples and recomendations of TensorFlow Lite, availables on its website\footnote{\url{https://www.tensorflow.org/lite/?hl=es_419}}.

The image generator application was developed using the video game engine Unity. The three-dimensional models used in this application have been sourced from CGTrader\footnote{\url{https://www.cgtrader.com/}}, Free3D\footnote{\url{https://free3d.com/es/}}, TurboSquid\footnote{\url{https://www.turbosquid.com/}}, 3DModelHeaven\footnote{\url{https://3dmodelhaven.com/}} and Unity Asset Store\footnote{\url{https://assetstore.unity.com/}}.

Lastly, for the project's version control it has been used GitHub and for the project report development the \texis template in TexMaker editor.

The entire process has been developed with a Windows device and the aplication tests with an Android mobile device.



%-------------------------------------------------------------------
\section{Work plan}
%-------------------------------------------------------------------
\label{cap1b:sec:work-plan}

The workcourse is divided into three different parts: the image generation, training the neural network and development the mobile object identification application.

The first one is focus on the develop of an application that loads 3D models and makes numerous images to each one, changing their position, rotation and background to obtain diversity in the images. The models are organised by their material so the images generated must be arranged the same way.
The development of the application is divided into two iterations, the first one is where all the main functionalities are developed and simultaneously verified its performance. Meanwhile, in the second iteration takes place the generation of the dataset.

Once the images are generated it is time for the neural network traning, which is the second part of the project. To develop this it's necessary to investigate the different types of neural networks and choose the most appropriate option to be used in a mobile application. In addition, it will be necessary to obtain some datasets of real images of the materials selected in order to make tests and comparisons of the accuracy in different cases.

Lastly, takes place the development of the mobile devices application, which uses the trained model to identify the material of the object that is being focused on with the camera. Among the material, it's also indicated how it must be recycled.
At this point in the coursework various tests must be done to verify the performance of the application. To do so, the application accuracy of different trained models will be compared using diverse everyday objects.




%-------------------------------------------------------------------
%\section*{\NotasBibliograficas}
%-------------------------------------------------------------------
%\TocNotasBibliograficas

%Citamos algo para que aparezca en la bibliograf�a\ldots
%\citep{ldesc2e}

%\medskip


%-------------------------------------------------------------------
%\section*{\ProximoCapitulo}
%-------------------------------------------------------------------
%\TocProximoCapitulo