%---------------------------------------------------------------------
%
%                          Cap�tulo 6
%
%---------------------------------------------------------------------

\chapter{Conclusiones y Trabajo Futuro}



%-------------------------------------------------------------------
\section{Conclusiones}
%-------------------------------------------------------------------
\label{cap6:sec:conclusiones}
\todo[Conclusiones de la generaci�n de im�genes, si es �til si ahorra tiempo.
�Encuesta sobre si la aplicaci�n ser�a �til? Para escribir algo sobre que s� es algo que la gente considere necesario. Y qu� opciones prefieren.]


%-------------------------------------------------------------------
\section{Trabajo Futuro}
%-------------------------------------------------------------------
\label{cap6:sec:trabajo-futuro}
\todo[A�adir m�s materiales, m�s objetos, hacer una interfaz propia, accesibilidad, modelos m�s realistas para el entrenamiento?, Mejor iluminaci�n?]


%-------------------------------------------------------------------
%\section*{\NotasBibliograficas}
%-------------------------------------------------------------------
%\TocNotasBibliograficas

%Citamos algo para que aparezca en la bibliograf�a\ldots
%\citep{ldesc2e}

%\medskip

%Y tambi�n ponemos el acr�nimo \ac{CVS} para que no cruja.

%Ten en cuenta que si no quieres acr�nimos (o no quieres que te falle la compilaci�n en ``release'' mientras no tengas ninguno) basta con que no definas la constante \verb+\acronimosEnRelease+ (en \texttt{config.tex}).


%-------------------------------------------------------------------
%\section*{\ProximoCapitulo}
%-------------------------------------------------------------------
%\TocProximoCapitulo


% Variable local para emacs, para  que encuentre el fichero maestro de
% compilaci�n y funcionen mejor algunas teclas r�pidas de AucTeX
%%%
%%% Local Variables:
%%% mode: latex
%%% TeX-master: "../Tesis.tex"
%%% End:
