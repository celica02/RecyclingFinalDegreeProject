%---------------------------------------------------------------------
%
%                          Capítulo 4
%
%---------------------------------------------------------------------

\chapter{Red Neuronal}

\begin{FraseCelebre}
\begin{Frase}
...
\end{Frase}
\begin{Fuente}
...
\end{Fuente}
\end{FraseCelebre}

\begin{resumen}
...
\end{resumen}


%-------------------------------------------------------------------
\section{Tensorflow}
%-------------------------------------------------------------------
\label{cap4:sec:tensorflow}


%-------------------------------------------------------------------
\section{Tensorflow Lite}
%------------------------------------------------------------------
\label{cap4:sec:tensorflow-lite}

Para la red neuronal se utiliza TensorFlow Lite, un framework de código abierto de aprendizaje profundo para dispositivos móviles. 
TensorFlow Lite Permite ejecutar modelos de aprendizaje profundo en dispositivos móviles. A través de él se puede utilizar modelos entrenados en una computadora en un dispositivo móvil sin necesidad de utilizar un servidor. Utiliza MobileNet, que está diseñada y optimizada para imágenes en móviles, incluyendo detección y clasificación de objetos, detección de caras y reconocimiento de **lugares**

modelo de TensorFlow Lite (un formato FlatBuffer optimizado que se puede identificar mediante la extensión de archivo .tflite
TensorFlow Lite incorpora (esto significa que se ejecuta en el dispositivo móvil) TensorFlow en dispositivos móviles. Anunciado en 2017, el paquete de software TFLite está diseñado específicamente para el desarrollo móvil. TensorFlow Lite “Micro” es, por otro lado, una versión especialmente para microcontroladores, que se fusionó recientemente con uTensor de ARM.


%-------------------------------------------------------------------
\section{Entrenamiento}
%-------------------------------------------------------------------
\label{cap4:sec:entrenamiento}

A partir del blog de TensorFlow y sus ejemplos se ha generado un script sencillo de carga de imágenes, separación de etiquetas y entrenamiento con todo esto que genera el modelo entrenado, un .tflite ya que se utiliza la extensión de TensorFlow: TensorFlow Lite, para dispositivos móviles. Y un documento de texto plano con las distintas etiquetas disponibles.
Dicho script tiene dos maneras de utilizar las imágenes, la primera es que carga todas las imágenes de una carpeta, cada tipo de imagen estará separada en subcarpetas las cuales serán las distintas etiquetas. Una vez cargadas todas las imágenes se separan en las de entrenamiento y las de test, siendo un 90\% para lo primero.
La otra opción es para comprobar qué tal detectaba los objeto si se entrenaba con las imágenes generadas (spoiler: sale mal), se cargan primero las imágenes de entrenamiento de la carpeta correspondiente, la cual tiene la separación de materiales en subcarpetas; y después se cargan las imágenes de test, que serán imágenes reales de objetos de ese material, también estarán organizadas en subcarpetas según el material correspondiente, y debe haber tantas como en las de entrenamiento. Despues de entrenar la red se comprueba con cuanta precisión detecta los objetos en las imágenes de test.

%-------------------------------------------------------------------
\section{Comparación}
%-------------------------------------------------------------------
\label{cap4:sec:comparación}


...

%-------------------------------------------------------------------
\section*{\NotasBibliograficas}
%-------------------------------------------------------------------
\TocNotasBibliograficas

%Citamos algo para que aparezca en la bibliografía\ldots
%\citep{ldesc2e}

%\medskip

%Y también ponemos el acrónimo \ac{CVS} para que no cruja.

%Ten en cuenta que si no quieres acrónimos (o no quieres que te falle la compilación en ``release'' mientras no tengas ninguno) basta con que no definas la constante \verb+\acronimosEnRelease+ (en \texttt{config.tex}).


%-------------------------------------------------------------------
\section*{\ProximoCapitulo}
%-------------------------------------------------------------------
\TocProximoCapitulo

...

% Variable local para emacs, para  que encuentre el fichero maestro de
% compilación y funcionen mejor algunas teclas rápidas de AucTeX
%%%
%%% Local Variables:
%%% mode: latex
%%% TeX-master: "../Tesis.tex"
%%% End:
