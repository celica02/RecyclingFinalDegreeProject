%---------------------------------------------------------------------
%
%                          Capítulo 2
%
%---------------------------------------------------------------------

\chapter{Estado del Arte}

\begin{FraseCelebre}
\begin{Frase}
...
\end{Frase}
\begin{Fuente}
...
\end{Fuente}
\end{FraseCelebre}

\begin{resumen}
...
\end{resumen}


%-------------------------------------------------------------------
\section{Aplicaciones de Reciclaje}
%-------------------------------------------------------------------
\label{cap2:sec:aplicaciones-reciclaje}

Debido a la gran cantidad de materiales disitntos que se encuentran en los resudios de una persona normal de un país primer mundista, y la cantidad de residuos que se generan, ha surgido una necesidad (para aquellos que se preocupan algo por el planeta) de ayuda para reciclar correctamente todos estos residuos. 

Esta ayuda ha llegado, sobre todo, en forma de aplicación móvil. Con lo que se ha generalizado el uso de estos en la población por su comodidad y su gran alcance es entendible que si esta ayuda puede estar tan a mano y tan accesible como es un teléfono móvil, mejor. 

En España, para empezar, existe la aplicación de "AIRE Asistente Inteligente de Reciclaje" disponible tanto para iOS\footnote{https://apps.apple.com/es/app/air-e/id1442582214} como Android\footnote{https://play.google.com/store/apps/details?id=com.ecoembes.aire&hl=en} aplicación publicada por "ecoembes", empresa encargada del reciclaje de los residuos de los contenedores amarillo y azul en España. 

Carrefour, por ejemplo, también cuenta con una aplicación de ayuda al reciclaje\footnote{https://www.reciclaya.app/es/como-funciona}, esta es a partir del código de barras del ticket de tu compra.



%-------------------------------------------------------------------
\section{Redes Neuronales}
%-------------------------------------------------------------------
\label{cap2:sec:redes-neuronales}


%-------------------------------------------------------------------
\section{Identificación de Objetos e Imágenes}
%-------------------------------------------------------------------
\label{cap2:sec:identificacion}


%-------------------------------------------------------------------
\section{Generación de Imágenes}
%-------------------------------------------------------------------
\label{cap2:sec:generacion}


...

%-------------------------------------------------------------------
\section*{\NotasBibliograficas}
%-------------------------------------------------------------------
\TocNotasBibliograficas

Citamos algo para que aparezca en la bibliografía\ldots
\citep{ldesc2e}

\medskip

Y también ponemos el acrónimo \ac{CVS} para que no cruja.

Ten en cuenta que si no quieres acrónimos (o no quieres que te falle la compilación en ``release'' mientras no tengas ninguno) basta con que no definas la constante \verb+\acronimosEnRelease+ (en \texttt{config.tex}).


%-------------------------------------------------------------------
\section*{\ProximoCapitulo}
%-------------------------------------------------------------------
\TocProximoCapitulo

...

% Variable local para emacs, para  que encuentre el fichero maestro de
% compilación y funcionen mejor algunas teclas rápidas de AucTeX
%%%
%%% Local Variables:
%%% mode: latex
%%% TeX-master: "../Tesis.tex"
%%% End:
