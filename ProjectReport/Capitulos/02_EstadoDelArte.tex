%---------------------------------------------------------------------
%
%                          Cap�tulo 2
%
%---------------------------------------------------------------------

\chapter{Estado del arte}



%-------------------------------------------------------------------
\section{Aplicaciones de reciclaje}
%-------------------------------------------------------------------
\label{cap2:sec:aplicaciones-reciclaje}

Debido a los distintos materiales que se encuentran en los residuos del d�a a d�a de la mayor�a de las personas y la cantidad de estos que se generan, ha surgido una necesidad de ayuda para reciclar correctamente todos estos residuos. 

Esta ayuda ha llegado, sobre todo, en forma de aplicaci�n m�vil. Con la generalizaci�n del uso de los dispositivos m�viles en la poblaci�n favorece a que esta sea la mejor forma de encontrar esta ayuda.

En Espa�a, para empezar, existe la aplicaci�n de ``AIRE Asistente Inteligente de Reciclaje'' disponible tanto para iOS\footnote{\url{https://apps.apple.com/es/app/air-e/id1442582214}} como Android\footnote{\url{https://play.google.com/store/apps/details?id=com.ecoembes.aire&hl=en}} aplicaci�n publicada por ``ecoembes'', empresa encargada del reciclaje de los residuos de los contenedores amarillo y azul en Espa�a. 

Carrefour, por ejemplo, tambi�n cuenta con una aplicaci�n de ayuda para reciclaje\footnote{\url{https://www.reciclaya.app/es/como-funciona}}, esta a partir del c�digo de barras del ticket de la compra que se haya hecho indica c�mo reciclar los envases de los productos de algunas marcas.

En el lado internacional se encuentran aplicaciones como RecycleRight, Brisbane Bin and Recycling, Grow Recycling o Recycle Coach, aunque distintas entre ellas todas estas aplicaciones cuentan con un factor de explicaci�n y aprendizaje sobre reciclaje para el usuario.


%-------------------------------------------------------------------
\section{Redes neuronales}
%-------------------------------------------------------------------
\label{cap2:sec:redes-neuronales}
Redes neuronales m�s usadas. En qu� se usan. Aplicaciones que las utilicen.

Las redes neuronales artificiales vienen inspiradas por la funcionalidad sofisticada del cerebro humano donde miles de millones de neuronas se encuentran interconectadas y procesan informaci�n de manera paralela.
Una red neuronal artificial consta de una capa de entrada de neuronas, una o varias capas ocultas de neuronas y una capa final de neuronas de salida. En la figura\ref{fig:RN} 
se muestra la estructura habitual de una red neuronal artificial. Se muestra mediante l�neas la conexi�n de las neuronas. Cada una de estas conexiones se asocia con un valor num�rico llamado peso. \cite{Wang2003};

\figura{RedNeuronal.png}{width=.5\textwidth}{fig:RN}{Estructura de una red neuronal artificial.}

%-------------------------------------------------------------------
\section{Generaci�n de im�genes}
%-------------------------------------------------------------------
\label{cap2:sec:generacion}
Esta parte, de generaci�n de im�genes, se encuentra menos explorada que las dem�s del cap�tulo. A�n as� encontramos numerosos acercamientos desde distintas �reas de investigaci�n.

En este �mbito se encuentran las Redes Generativas Adversativas, que se relaciona con el tratamiento de im�genes. Este modelo consta de dos redes neuronales denominadas generador y discriminador, y el objetivo es generar datos similares a los que se han usado para el entrenamiento.
La red generador, como su nombre indica, es la encargada de generar datos del tipo de los del entrenamiento. Por otro lado, la red discriminador distingue entre los datos reales que se le proporcionan y los generados por la red anterior.
Esto fue propuesto en 2014 en el art�culo ``Generative Adversarial Networks'' \footnote{\url{https://arxiv.org/abs/1406.2661}}.

%-------------------------------------------------------------------
\section{Identificaci�n de objetos e im�genes}
%-------------------------------------------------------------------
\label{cap2:sec:identificacion}
Google Lens, Google Photos, Microsoft  Office -> ejemplos m�s conocidos.


%-------------------------------------------------------------------
%\section*{\NotasBibliograficas}
%-------------------------------------------------------------------
%\TocNotasBibliograficas

%Citamos algo para que aparezca en la bibliograf�a\ldots
%\citep{ldesc2e}

%\medskip

%Y tambi�n ponemos el acr�nimo \ac{CVS} para que no cruja.

%Ten en cuenta que si no quieres acr�nimos (o no quieres que te falle la compilaci�n en ``release'' mientras no tengas ninguno) basta con que no definas la constante \verb+\acronimosEnRelease+ (en \texttt{config.tex}).


%-------------------------------------------------------------------
%\section*{\ProximoCapitulo}
%-------------------------------------------------------------------
%\TocProximoCapitulo
