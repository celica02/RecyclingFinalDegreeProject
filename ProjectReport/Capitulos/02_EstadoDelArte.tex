%---------------------------------------------------------------------
%
%                          Cap�tulo 2
%
%---------------------------------------------------------------------

\chapter{Estado del Arte}



%-------------------------------------------------------------------
\section{Aplicaciones de Reciclaje}
%-------------------------------------------------------------------
\label{cap2:sec:aplicaciones-reciclaje}

Debido a los distintos materiales que se encuentran en los residuos del d�a a d�a de la mayor�a de las personas y la cantidad de estos que se generan, ha surgido una necesidad de ayuda para reciclar correctamente todos estos residuos. 

Esta ayuda ha llegado, sobre todo, en forma de aplicaci�n m�vil. Con la generalizaci�n del uso de los dispositivos m�viles en la poblaci�n favorece a que esta sea la mejor forma de encontrar esta ayuda.

En Espa�a, para empezar, existe la aplicaci�n de ``AIRE Asistente Inteligente de Reciclaje'' disponible tanto para iOS\footnote{\url{https://apps.apple.com/es/app/air-e/id1442582214}} como Android\footnote{\url{https://play.google.com/store/apps/details?id=com.ecoembes.aire&hl=en}} aplicaci�n publicada por ``ecoembes'', empresa encargada del reciclaje de los residuos de los contenedores amarillo y azul en Espa�a. 

Carrefour, por ejemplo, tambi�n cuenta con una aplicaci�n de ayuda para reciclaje\footnote{\url{https://www.reciclaya.app/es/como-funciona}}, esta a partir del c�digo de barras del ticket de la compra que se haya hecho indica c�mo reciclar los envases de los productos de algunas marcas.

En el lado internacional se encuentran aplicaciones como RecycleRight, Brisbane Bin and Recycling, Grow Recycling o Recycle Coach, aunque distintas entre ellas todas estas aplicaciones cuentan con un factor de explicaci�n y aprendizaje sobre reciclaje para el usuario.


%-------------------------------------------------------------------
\section{Redes Neuronales}
%-------------------------------------------------------------------
\label{cap2:sec:redes-neuronales}
Redes neuronales m�s usadas. En qu� se usan. Aplicaciones que las utilicen.

%-------------------------------------------------------------------
\section{Generaci�n de Im�genes}
%-------------------------------------------------------------------
\label{cap2:sec:generacion}
Redes Neuronales Generativas

%-------------------------------------------------------------------
\section{Identificaci�n de Objetos e Im�genes}
%-------------------------------------------------------------------
\label{cap2:sec:identificacion}
Google Lens, Google Photos, Microsoft  Office -> ejemplos m�s conocidos.


%-------------------------------------------------------------------
\section*{\NotasBibliograficas}
%-------------------------------------------------------------------
%\TocNotasBibliograficas

%Citamos algo para que aparezca en la bibliograf�a\ldots
%\citep{ldesc2e}

%\medskip

%Y tambi�n ponemos el acr�nimo \ac{CVS} para que no cruja.

%Ten en cuenta que si no quieres acr�nimos (o no quieres que te falle la compilaci�n en ``release'' mientras no tengas ninguno) basta con que no definas la constante \verb+\acronimosEnRelease+ (en \texttt{config.tex}).


%-------------------------------------------------------------------
%\section*{\ProximoCapitulo}
%-------------------------------------------------------------------
%\TocProximoCapitulo
