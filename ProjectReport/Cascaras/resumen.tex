%---------------------------------------------------------------------
%
%                      resumen.tex
%
%---------------------------------------------------------------------
%
% Contiene el cap�tulo del resumen.
%
% Se crea como un cap�tulo sin numeraci�n.
%
%---------------------------------------------------------------------

\chapter{Resumen}
\cabeceraEspecial{Resumen}


Con el incremento de los envases de un solo uso y el aumento de los tipos de materiales, se ha observado un crecimiento masivo en los residuos, y estos, al no ser gestionados adecuadamente, han provocado un gran impacto medioambiental. Para hacerle frente a este problema una acci�n sencilla pero efectiva es separar los residuos en el origen. No obstante, con la diversidad de materiales existentes actualmente puede resultar complicado saber c�mo desecharlos todos correctamente. Este trabajo de fin de grado busca ofrecer, mediante una aplicaci�n m�vil, una peque�a ayuda para solventar estas dudas a personas que lo puedan necesitar.

Para ello, se ha querido aprovechar el gran avance que se ha vivido en visi�n computacional durante los �ltimos a�os y desarrollar una aplicaci�n de identificaci�n de objetos. Para esto se ha utilizado TensorFlow Lite, cuyas librer�as facilitan tanto el entrenamiento de la red neuronal y la generaci�n del modelo, como su importaci�n en aplicaciones m�viles. Asimismo, para facilitar la tarea de obtenci�n de \textit{datasets} para el entrenamiento, se ha desarrollado una aplicaci�n de generaci�n de im�genes sint�ticas, las cuales se generan a partir de modelos tridimensionales y son utilizadas posteriormente para el entrenamiento de la red.





\textbf{Palabras clave:} identificaci�n de objetos, red neuronal, TensorFlow Lite, generaci�n de im�genes, \textit{dataset}, reciclaje, material, aplicaci�n m�vil, modelo tridimensional.
\endinput
% Variable local para emacs, para  que encuentre el fichero maestro de
% compilaci�n y funcionen mejor algunas teclas r�pidas de AucTeX
%%%
%%% Local Variables:
%%% mode: latex
%%% TeX-master: "../Tesis.tex"
%%% End:

