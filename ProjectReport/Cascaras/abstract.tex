%---------------------------------------------------------------------
%
%                      abstract.tex
%
%---------------------------------------------------------------------
%
% Contiene el cap�tulo del resumen en ingl�s.
%
% Se crea como un cap�tulo sin numeraci�n.
%
%---------------------------------------------------------------------

\chapter{Abstract}
\cabeceraEspecial{Abstract}

With the increase of single-use packaging and the different types of materials, there has been a massive growth in waste. Inappropiate management of these wastes have caused a huge environmental impact. To deal with this problem, one simple but effective action is to separate the waste at the source point. However, with the diversity of materials available today it can be difficult to know how to correctly dispose all of them. This final degree project seeks to offer help to solve the doubts through a mobile application.

To do so, we wanted to take advantage of the great progress that has been made in computer vision in recent years and develop an application that identifies objects. To achieve this, TensorFlow Lite has been used, whose libraries provides an easy way for training the needed neural network and generate its model, as well as its import into mobile applications. Likewise, to ease the task of obtaining the necessary datasets for training, an application that generates synthetic images from three-dimensional models has been developed. These images are subsequently used for the neuronal network training.

As a result we obtain a mobile application which identifies different materials trained using synthetic images.


\medskip
\medskip
\medskip
\medskip
\textbf{Key words:} recycling, object identification, neuronal network, TensorFlow Lite, image generation, computer vision, dataset, mobile application.


\medskip
\medskip
\medskip
\medskip
\medskip
\medskip
All materials can be found in the GitHub repository \url{https://github.com/celica02/RecyclingFinalDegreeProject}
