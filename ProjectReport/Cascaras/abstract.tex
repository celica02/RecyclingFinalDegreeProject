%---------------------------------------------------------------------
%
%                      abstract.tex
%
%---------------------------------------------------------------------
%
% Contiene el cap�tulo del resumen en ingl�s.
%
% Se crea como un cap�tulo sin numeraci�n.
%
%---------------------------------------------------------------------

\chapter{Abstract}
\cabeceraEspecial{Abstract}

With the increase of single-use packaging and the different types of materials, there has been a massive growth in waste. An inappropiate managed of these wastes have caused a huge environmental impact. To deal with this problem, a simple but effective action is to separate the waste at the source point. However, with the diversity of materials available today it can be difficult to know how to dispose correctly all of them. This final degree project seeks to offer help to solve the doubts through a mobile application.

To do so, we wanted to take advantage of the great progress that has been made in computer vision in recent years and develop an application that identifies objects. To achieve this it has been used TensorFlow Lite, whose libraries provides an easy way for training the neural network and generate the model, as well as its import into mobile applications. Likewise, to ease the task of obtaining the necessary datasets for training, an application that generates synthetic images from three-dimensional models has been developed. This images are subsequently used for the neuronal network training.

\textbf{Key words:} object identification, neuronal network, TensorFlow Lite, image generation, datsset, recycling, material, mobile application, three-dimensional model.